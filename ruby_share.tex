\documentclass[11pt]{beamer}
\usetheme{Luebeck}
\usecolortheme{beaver}

\usepackage{minted}
\usepackage{graphicx}

\begin{document}
\title{Ruby Metaprogramming}
\author{Zhang Yuning}
\date{\today}
\begin{frame}
\titlepage
\end{frame}

\begin{frame}[fragile]
\frametitle{Hello, world!}
\begin{minted}[linenos=true]{ruby}
# myapp.rb
require 'sinatra'

get '/' do
  'Hello world!'
end
\end{minted}
\end{frame}

\begin{frame}
\frametitle{Ruby's Feature}
\begin{itemize}
\item Completely object-oriented
\item Blocks
\item Metaprograming
\item Dynamic typing
\item Garbage collection
\item Rails Girls
\end{itemize}
\end{frame}

\begin{frame}[fragile]
\frametitle{Ruby Basics}
In ruby there is only expressions:
\begin{minted}[linenos=true]{ruby}
winston = if 2 + 2 == 5
            'Ignorance is strength'
          else
            'Freedom'
          end # => 'Freedom'
winston # => 'Freedom'
\end{minted}
\end{frame}

\begin{frame}[fragile]
\frametitle{Ruby Basics}
Symbols, like in Lisp:
\begin{minted}{ruby}
  :symbol
\end{minted}
Used mainly in metaprogramming methods
\begin{minted}{ruby}
  attr_reader :length
\end{minted}
\end{frame}

\begin{frame}[fragile]
\frametitle{Ruby Basics - Naming convention}
predicates' name should end with ?
\begin{minted}{ruby}
  block_given?
  empty?
\end{minted}
`impure' methods' name should end with !
\begin{minted}{ruby}
  reverse!
  reverse
\end{minted}
\end{frame}

\begin{frame}[fragile]
\frametitle{Ruby Basics}
Blocks are a important feature of Ruby:
\begin{minted}[linenos=true]{ruby}
[1, 2, 3].each do |x|
  puts x
end
\end{minted}
We will meet it often.
\end{frame}

\begin{frame}[fragile]
\frametitle{Ruby Basics}
every method can be given a block and call it with yield
\begin{minted}[linenos=true]{ruby}
def call_with_42
  if block_given?
    yield 42
  end
end
call_with_42 {|x| puts x}
\end{minted}
\end{frame}

\begin{frame}[fragile]
\frametitle{Everything is object}
Fixnum is object:
\begin{minted}{ruby}
3.times do
  puts 'quark'
end
# even
3.days.ago
\end{minted}
\end{frame}

\begin{frame}[fragile]
\frametitle{Everything is object}
nil is also object:
\begin{minted}[linenos=true]{ruby}
# Note that built-in Classes can be modified
class Object
  def try
    if block_given?
      yield self
    end
  end
end

class NilClass
  def try
    nil
  end
end
\end{minted}
\end{frame}

\begin{frame}[fragile]
\frametitle{Everything is object}
Usage:
\begin{minted}{ruby}
nil.try{|x| x + 1} # => nil
2.try{|x| x + 1} # => 3
\end{minted}
\begin{minted}{haskell}
Nothing >>= (\x -> Just (x + 1)) # => Nothing
Just 2 >>= (\x -> Just (x + 1)) # => Just 3
\end{minted}
\end{frame}

\begin{frame}[fragile]
\frametitle{Everything is object}
Classes are objects, too!
\begin{minted}[linenos=true]{ruby}
class A
end
A.class # => Class
Class.class # => Class
\end{minted}
\end{frame}

\begin{frame}[fragile]
\frametitle{Everything is object}
From a linguist's point of view, Blocks are not objects.
But there are three ways to convert them to objects.
(Note: there are subtle differences between each other)
\begin{minted}{ruby}
Proc.new {|x| x + 1}
lambda {|x| x + 1}
-> x { x + 1 } # introduce in ruby 1.9
\end{minted}
\begin{minted}{ruby}
p = Proc.new {|x| x + 1}
p.(1)
p[1]
p.call 1
\end{minted}
Note that object is first-class. So...
\end{frame}

\begin{frame}[fragile]
\frametitle{Duck Typing}
\begin{quotation}
But there`s an old American phrase about if it walks like a duck and quacks like a duck and so 
forth, it`s a duck.
\rightline{--- Mike Wallace}
\end{quotation}
\begin{minted}[linenos=true]{ruby}
# taken from Sinatra (modified)
if boom.respond_to? :http_status
  status(boom.http_status)
elsif boom.respond_to?:code and boom.code.between? 400,599
  status(boom.code)
else
  status(500)
end
\end{minted}
\end{frame}

\begin{frame}[fragile]
\frametitle{Singleton Method}
\begin{minted}[linenos=true]{ruby}
class A
  def hello
    'hello'
  end
end
a1 = A.new
a2 = A.new
def a1.bye
  'bye'
end
a1.hello # => 'hello'
a1.bye # => 'bye'
a2.hello # => 'hello'
a2.respond_to? :bye # => false
\end{minted}
\end{frame}

\begin{frame}[fragile]
\frametitle{Method Lookup}
\begin{figure}[htbp]
    \includegraphics[scale=0.4]{method_lookup.png}
\end{figure}
\end{frame}

\begin{frame}[fragile]
\frametitle{Class definition}
Class definition in Ruby just changes the environment.
You can invoke method in them.
\begin{minted}{ruby}
class A
  puts 'hello, world'
end
\end{minted}
\end{frame}

\begin{frame}[fragile]
\frametitle{define\_method \& method\_missing}
    The html example.
\end{frame}

\begin{frame}[fragile]
\frametitle{Hooks}
\begin{minted}[linenos=true, breaklines=true]{ruby}
at_exit { Application.run! if $!.nil? && Application.run? }

class Base
  def inherited(subclass)
    subclass.reset!
    subclass.set :app_file, caller_files.first unless subclass.app_file?
  end
end
\end{minted}
\end{frame}

\begin{frame}[fragile]
\frametitle{Wrap up - The logger decorator}
\begin{minted}[linenos=true]{ruby}
class A
  include Logger

  def f
    puts 'hello, world!'
  end

  add_logger :f
end
# output:
# f started
# hello, world!
# f finished
\end{minted}
\end{frame}
\end{document}
